\documentclass[12pt]{article}

% Packages
\usepackage{amsmath, amssymb}
\usepackage{graphicx}
\usepackage{geometry}
\usepackage[backend=biber, style=apa]{biblatex} % Use biblatex with biber
\usepackage{wrapfig}
\usepackage{appendix} % For appendix management
\usepackage{booktabs} % For better table lines
\usepackage{color,soul}
\usepackage{hyperref}
%make links look nice

\graphicspath{{../matlab_base/output/}}
% Page layout
\geometry{a4paper, margin=1in}

% Bibliography file
\addbibresource{references.bib} % Specify the .bib file

% Title and Author
\title{Title of the Paper}
\author{Author Name}
\date{\today}

% turning off some linting errors
% chktex-file 3

% ------------------------------------------------------------------------------
% Document
\begin{document}

\maketitle
% 1. Executive Summary
\section*{1. Executive Summary (max half page)}
% Your summary here.

% 2. Introduction
\section*{2. Introduction (about one page)}
\subsection*{(a) Cause and impact of economic fluctuations}
% Content here.

\subsection*{(b) Consequences for economy and why policy should act}
% Content here.

\subsection*{(c) Brief overview of suggested policy}
% Content here.



% ------------------------------------------------------------------------
% THEO ------------------------
% ------------------------------------------------------------------------
\newpage
% 3. Bird’s Eye View of the Model
\section*{3. Bird’s Eye View of the Model (about two pages)}



\subsection*{(a) Outline of the Model Structure (graphical and verbal, no equations)}
% Content here (include figure environment if needed).
The Dynamic Stochastic General Equilibrium (DSGE) Model used for this policy analysis features
three representative agents, namely a representative household, an intermediate goods firm and final goods firm, as well as the \hl{government}.\footnote{Jakub: I would argue that government is not an agent --- it doesn't maximize anything, it just follows a rule (same as the CB), so maybe I wouldn't call it that (instead write ``The model also includes gov + cb''?)} 

The representative household aims to maximize its intertemporal utility, i.e.~well-being, across all periods.\ \hl{It derives utility from consumption and its utility is reduced by working}\footnote{Jakub: I would re-write to ``It's utility is increasing in consumption and decreasing in hours worked''}. 
To maximize its objective function it chooses consumption, labour effort and investment in capital, in every period. %Ct,Lt,Xt,Mt+1,Bt+1,Kt+1
Furthermore, it \hl{makes choices that affect the next period, how much money holdings to save, which determines the money available in the next period}.\footnote{Jakub:``HH's choices today impact its outcomes in the future. Therefore, when choosing its money savings for the next period\ \dots''} As well as how many bonds to pay at price $q_t$ which pay off in the next period.\footnote{Should I say here it also chooses $K_{t+1}$ I think that is covered by $X_t$} 

\footnote{I would consider putting this paragraph before the previous one because the CiA constraint means that the HH has to save money for the future --- better continuity}When trying to maximize its utility the representative household faces two constraints. First, to consume it must have cash available beforehand, i.e.\ only money saved in the last period, at 0\% interest compared to $\frac{1}{q} -1$ from the bonds, can be used to buy consumption goods. 
The second constraint is a common budget constraint, which simply requires that income be larger than or equal to expenses.  \hl{The households value utility derived today more highly than that derived in a future period. Thus, future utility is discounted.}\footnote{I would re-write this into one sentence ``HH is assumed to discount future utility in to reflect the fact that \dots''. }

The intermediate goods firm as well as the final goods firm maximize their profit. The revenue of the intermediate firm stem from selling their intermediate good to the final goods firm. The cost incurred consist of the nominal wage paid to labour, rent paid to capital, both of which are provided by the household. It is bound by a Cobb-Douglas function with a price stickiness friction, and the supply of all intermediate goods must be met by demand for that good from the final good firm.  As just explained the final good firm buys the intermediate goods. The firm's input are substitutable to some degree. 

The government has fiscal and monetary policy at their disposal, but cannot combine the two. 



\subsection*{(b) Policy Structure of the Model (main equations)}
The government has fiscal and monetary policy at their disposal, but cannot combine the two. Our baseline scenario assumes that the government
follows a Taylor-like monetary policy rule, which is a function of the inflation rate and determines the change in money supply. If the inflation rate is above the target level, the money supply is decreased and vice versa.\footnote{Easier to see if minus in front of theta is removed and fraction flipped} The strength of the change in money supply relative to the deviation of inflation is given by the parameter $\theta_{pi}$. The monetary policy rule is given by:
The Taylor rule of monetary policy is given by:
\begin{align}
    \frac{1 + r_{B,t}}{1 + \bar{r}_B} = \left( \frac{1 + r_{B,t-1}}{1 + \bar{r}_B} \right)^{\theta_R} \left( \frac{1 + \pi_t}{1 + \pi^*} \right)^{(1 - \theta_R)\theta_{pi}}
e^{M_t}
\end{align}

Next par.\footnote{Change the policy to the baseline one. Explain here how the policy affects the economy.}

In the counterfactual scenarios we first consider a change of $\theta_\pi$, i.e.~the responsiveness of the money supply to inflation from its target. Then we include a secondary term, meaning that the central bank not only cares about inflation but also employment levels, similar to the dual mandate of the US.\@ As a third monetary policy rule we consider including the marginal cost in the Taylor-like rule.

The fiscal policy rule works similarly\footnote{Should we include a fiscal policy rule as alternative policy?}, but instead of changing the money supply, the government adjusts a variable tax surcharge ($\tau_t$). This surcharge is added to the individual tax rates for labour income, capital income and consumption. This surcharge is a function of the level of government debt ($l_{g,t}$) (measured in \% of GDP) and the steady-state target level of government debt ($l^{\star}_{g}$). The government reacts to the difference between the two levels of government debt by adjusting the surcharge. If the debt level is above its steady state target the surcharge is increased above its steady state level and vice versa. The incremental change of the surcharge is guided by a parameter $\lambda_{l g}$, which determines how responsive the government is to the deviation of government debt from its target. The fiscal policy rule is given by:
\begin{align}
    \tau_t - \tau^{\star} = \lambda_{l g} (l_{g,t} - l^{\star}_{g})
\end{align}

Next par.\footnote{Explain here how the policy affects the economy.}


% Content here (use align or equation environments).

\subsection*{(c) Calibration of the Model}
The model is calibrated to the Canadian economy in 1979, assuming that one period corresponds to a year. 
The discount factor $\beta$ is set to 0.96, inline with \textcite{someOilDemandSupply2023} and close to \textcite{corriganToTEMIIIBank2021}. 
The intertemporal consumption elasticity $\sigma$ is set to 2. Positive values indicate risk-aversion, whereas a negative value would correspond to risk-loving behaviour
\parencite{thimmeIntertemporalSubstitutionConsumption2017}.

Estimates for the inverse Frisch elasticity $\gamma$ differ starkly, between macro and micro estimates. Reasons for this discrepancy
include that micro studies use a sample which constitutes a subset of the population (e.g.\ primary earners, in the midst of their careers) whereas macro 
estimates try to copy behaviour of the entire population. Furthermore, micro estimates usually do not consider the extensive margin, i.e.\ the decision 
to work or not. Since the model attempts to capture macro trends we use a macro estimate of 3 \parencite{petermanReconcilingMicroMacro2016}. 

Money velocity $\nu$ is calculated, using Fisher's quantity equation, as the ratio of nominal GDP to the money stock (M1) (with data from: \textcite{bankofenglandCanadianDollarData2021}; 
\textcite{federalreservebankofminneapolisInflationCalculatorFederal}; \textcite{worldbankgroupWorldBankNational}; \textcite{bankofcanadaSelectedMonetaryAggregates})\footnote{GDP from World Bank in 2015 USD converted to 1979 USD using Federal Reserve Minneapolis, converted to CAD using historic annual average exchange rate from 
the Bank of England, Money Supply from Bank of Canada.} yielding 4.2. Thus, $\nu$ is set to 1/4.2 = 0.24. 

The capital depreciation rate $\delta$ is set to 0.1, matching the estimate of \textcite{statisticscanadaDepreciationRatesProductivity2007} and the calibration in \textcite{someOilDemandSupply2023} and \textcite{corriganToTEMIIIBank2021}.
The capital share of income $\alpha$ is set to 0.31~\parencite{fredst.louisShareLabourCompensation2021,feenstraNextGenerationPenn2015}. 
\footnote{The elasticity of substitution between differentiated goods, even though a parameter of the final goods firm is not really a production process parameter. 
Rather, it aggregates all goods into a single consumption good and thus is better understood as an elasticity of substitution between consuming differentiated goods.}. 

The three tax-rates are set in accordance with the average effective tax rates in Canada in 1980--1985. The labour income tax rate $\tau_L$ is set to 0.24, the capital income tax rate $\tau_K$ is set to 0.25 and the consumption tax rate $\tau_C$ is set to 0.13
\parencite{careyAverageEffectiveTax2000}. 

Price stickiness parameter $\kappa$ and the investment adjustment cost parameter $\phi_X$ are set to increase the model's fit to the data. As data, we use the deviation of inflation from its long-run (1961--2024) average ($\bar{\pi} =  3.72$)~\parencite{worldbank_inflation_ca}. To identify the deviation of GDP from its steady-state we use the one-sided Hodrick-Prescott filter, identifying the cyclical component of GDP \parencite{fred_gdp_per_capita_ca}.\@ We do not use a formal minimization process but rather visually inspect the best-fit.\footnote{Jakub: Citation added, delete if that's all I was supposed to do here}

% ------------------------------------------------------------------------
%  JAKUB -------------------------------
% ------------------------------------------------------------------------

% 4. Benchmark Model Analysis
\newpage
\section*{4. Benchmark Model Analysis (about two pages)}

As our scenario requires that the Central Bank adopts a money supply target aiming to achieve price stability, we adjust the benchmark model to a monetary policy rule with the following specification:\footnote{I don't get how the CB can set $M_t$ using $\pi_t$, CB sets M1 after prices are set but before HH go to market.}

\begin{equation}
    \frac{M_t}{M_{t-1}} = \Big(\frac{1+\pi_{t}}{1+ \bar \pi}\Big)^{\theta_{\pi}}
\end{equation}

Moreover, it seems natural, that the $MC$ shock should enter the model through the Phillips curve. That is because the curve represents the pricing decision rule of the intermediate companies, and so it will most accurately represent their reaction to an increase in marginal costs. Including our shock, the NK-Phillips curve is given by: 

\begin{equation}
    \frac{\kappa \pi_t (1 + \pi_t)}{1 - \frac{\kappa}{2} \pi_t^2}
    = \frac{1 - \frac{\rho}{mc_t \cdot e^{MC_s,t}}}{1 - \rho}
    + \mathbb{E}_t \beta_{t,t+1}
    \frac{Y_{t+1}}{Y_t}
    \frac{mc_{t+1} \cdot e^{MC_{s,t+1}}}{mc_t \cdot e^{MC_s,t}}
    \frac{\kappa \pi_{t+1} (1 + \pi_{t+1})^{2}}{1 - \frac{\kappa}{2} \pi_{t+1}^2}
\end{equation}

Where the periodic shock is expressed multiplicatively as $mc_t \cdot e^{MC_{s, t}}$ to ensure convenient estimation.

\subsection*{(a) How does the shock lead to economic fluctuations (baseline scenario)?}

These two assumptions lead to the fact that in the period following the shock, intermediate firms will adjust their prices and revise their hiring decisions. The economic intuition behind this is that an increase in marginal costs, decreases the markup of the firm which then chooses to adjust its prices. As adjustment is costly (due to $\kappa$), the firms will raise their prices less than the increase in marginal costs. In turn, higher prices, cause a decrease in demand for the aggregate product, therefore prompting the firms to decrease output. At the same time, the higher marginal costs imply that the companies choose to hire less labour and capital, as it becomes more costly. Finally, given the high prices and lower output, real wages and the real rental rate of capital fall.

Furthermore, the $mc_t \cdot e^{MC_{s, t}}$ feeds into firms' expectations about the future. When firms choose to set their prices for $t$, they trade off increasing their prices today and in the future. If the shock is expected to be persistent, firms may choose to increase their prices by less in $t$ and choose to keep increasing them slowly over time. 

The firm's pricing decision is also important for the future, as it is used as a benchmark relative to which they set their next period's prices. If prices are set high in period $t$,\ \dots\footnote{Does it increase the costs of raising the prices later on??? I don't think so}

Over time, as no more shocks happen, firms are able to arrive back at their target markup and therefore the economy is able to come back to the steady state inflation rate.\footnote{Expand on this part.}

\begin{figure}[!h]
    \caption{Impulse response functions of observable variables}\label{fig:main_baseline}
    \centering
    \includegraphics[width=0.7\textwidth]{main_baseline_policy.png}
\end{figure}

\begin{figure}[!h]
    \caption{Impulse response functions of observable variables}\label{fig:other_baseline}
    \centering
    \includegraphics[width=0.7\textwidth]{other_baseline_policy.png}
\end{figure}


Next par.\footnote{HERE DESCRIBE THE IRF'S AND DISCUSS MODEL LIMITATIONS --- E.G. CLOSED ECONOMY, NO EXCHANGE RATE ETC.}

% describe the short run as well as the path back to the steady state. 
\subsection*{(b) What is the idea behind the implemented policy (current policy scenario)?}

At the same time, in an effort to smooth out the path of the economy, the central bank will adjust the money supply to increase the purchasing power of the households and counter the shocks. By following the Taylor-like monetary policy rule, it will increase the money supply in order to match the inflation and marginal cost gaps of the economy in an effort to achieve price stability. As the bank intervention will increase the money stock of the households, it will allow it to consume more goods and therefore counter the drop in output that the economy experiences.\footnote{Think carefully about this, what is the ideal policy rule and how to actually best achieve price stability, what should the central bank target?}

\newpage
% 5. Counterfactual Policies
\section*{5. Counterfactual Policies (about two pages)}
\subsection*{(a) More or less of the same policy (different coefficients)?}


\begin{figure}[!h]
    \caption{Impulse response functions of observable variables}\label{fig:policy_par_variation}
    \centering
    \includegraphics[width=0.7\textwidth]{policy_par_variation.png}
\end{figure}
% Content here.

\subsection*{(b) Alternative policy (different targets)?}
% Content here.


\begin{figure}[!h]
    \caption{Impulse response functions of observable variables}\label{fig:policy_rule_variation}
    \centering
    \includegraphics[width=0.7\textwidth]{policy_rule_variation.png}
\end{figure}

% -------------------------------------------------------
%                       NOTES
% -------------------------------------------------------
% - try adding in the labour supply target instead of MC
% - think about implementing alternative inflation target (0.03)
% - think about alternative monetary policy rules 
% -  fiscal instead of monetary as a counter-factual?




% -------------------------------------------------------


% 6. Policy Recommendation and Discussion
\section*{6. Policy Recommendation and Discussion (about one page)}
\subsection*{(a) Give a recommendation on policy based on your simulations.}
% Content here.

\subsection*{(b) Should policy have taken a different path?}
% Content here.

\subsection*{(c) Do your results confirm or conflict with your perception about optimal policy? Discuss how.}
% Content here.

%references
\newpage %start a new page
\printbibliography{} % Print the bibliography

% Appendix 
\newpage
\appendix
\section{Appendix: Name of the Appendix}
\begin{table}[ht]
    \centering
    \caption{Parameter Calibration}\label{tab:parameters}
    \begin{tabular}{cll}
        \toprule
        Parameter & Description & Value  \\ \midrule
        %\hline \hline
        $\beta$ & Period discount factor  & 0.96  \\
        $\sigma$ & Intertemporal consumption elasticity  & 2  \\
        $\gamma$ & Inverse Frisch elasticity  & 3  \\
        $\nu$ & Money velocity  & 7  \\
        $\delta$ & Capital depreciation rate  & 0.1  \\
        $\phi_X$ & Investment adjustment cost parameter & \dots  \\
        $\alpha$ & Capital share of income & 0.31  \\
        $\rho$ & Elasticity of substitution for differentiated goods &  \dots  \\
        $\kappa$ & Price stickiness & 60 \\
        ${\tau_L}$ & Labour income tax & 0.24 \\
        ${\tau_K}$ & Capital income tax & 0.25 \\
        ${\tau_C}$ & Consumption tax & 0.13 \\
    \bottomrule
    \end{tabular}
\end{table}

\end{document}