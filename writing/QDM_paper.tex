\documentclass[12pt]{article}

% Packages
\usepackage{amsmath, amssymb}
\usepackage{graphicx}
\usepackage{geometry}
\usepackage[backend=biber, style=apa]{biblatex} % Use biblatex with biber
\usepackage{hyperref}

% Page layout
\geometry{a4paper, margin=1in}

% Bibliography file
\addbibresource{references.bib} % Specify the .bib file

% Title and Author
\title{Title of the Paper}
\author{Author Name}
\date{\today}

% ------------------------------------------------------------------------------
% Document
\begin{document}

\maketitle
%toc
%\tableofcontents

\section{Executive Summary (max half page)}

\section{Introduction (about one page) }
\subsection{(a) Cause and impact of economic fluctuations} 
\subsection{(b) Consequences for economy and why policy should act} 
\subsection{(c) Brief overview of suggested policy} 

\section{Bird’s Eye View of the Model (about two pages) }
\subsection{(a) Outline of the Model Structure (graphical and verbal, no equations) }
The Dynamic Stochastic General Equilibrium (DSGE) Model used for this policy analysis features
three representative agents, namely a representative household, intermediate goods firm and final goods firm, 
as well as the government. 

The reperesentative household aims to maximize its intertemporal utility, i.e. well-being 
across all periods. It derives utitlity from consumption and its utitltiy is reduced by working. To optimize the its objective
it chooses consumption, labour effort, investment in capital for the in every period. %Ct,Lt,Xt,Mt+1,Bt+1,Kt+1
Furthermore it makes choices that affect the next period,  how much money holdings to save, which determines the money available
in the next period. As well as how many bonds to pay at price $q_t$ which pay off in the next period. \emph{Should 
I say here it also chooses}$K_{t+1}$ \emph{I think that is covered by } $X_t$. When trying to maximize its utility
the reperesentative household faces two constraints. First, to consume it must have cash available beforehand, i.e. only money 
saved in the last period, at 0\% interest compared to $\frac{1}{q} -1$ from the bonds, can be used to buy consumption goods. 
The second constraint is a common budget constraint, which simply requires that income be larger or equal to expenses. 


\subsection{(b) Policy Structure of the Model (main equations)} 
\subsection{Calibration of the model} 

\section{Benchmark Model Analysis (about two pages)}
\subsection{(a) How does the shock lead to economic fluctuations (baseline scenario)? }
\subsection{(b) What is the idea behind the implemented policy (current policy scenario)? }

\section{Counterfactual Policies (about two pages)}
\subsection{(a) More or less of the same policy (different coefficients)? }
\subsection{(b) Alternative policy (different targets)? }

\section{Policy Recommendation and Discussion (about one page)}
\subsection{(a) Give a recommendation on policy based on your simulations. }
\subsection{(b) Should policy have taken a different path? }
\subsection{(c) Do your results confirm or conflict with your perception about optimal policy? Discuss how. }



% \printbibliography() % Print the bibliography

\section{Appendix}


\end{document}